\documentclass{libs/XJTLU_format}
% Inserting the preamble file with the packages
\input{libs/preamble.tex}
% Inserting the references file
\bibliography{libs/references.bib} 

% Title
\title[Defesa de Projeto de Graduação]{\normalsize\textbf{Desenvolvimento de Base de Dados para Treinamento de Redes Neurais de Reconhecimento de Voz
Através da Geração de Áudios com Resposta ao Impulso Simuladas por Técnicas de Data Augmentation}}
% Subtitle
%\subtitle{Creating Presentations}
% Author of the presentation
\author{Bruno Machado Afonso} 

% Institute's Name
\institute[- Escola Politécnica]{
    % email for contact
    \email{bruno.ma@poli.ufrj.br}
    \newline
    % Department Name
    \department{\scriptsize Departamento de Engenharia Eletrônica e de Computação - Escola Politécnica}
    \newline
    % University name
    \university{\scriptsize Universidade Federal do Rio de Janeiro}
}
% date of the presentation
\date{\today}


%%%%%%%%%%%%%%%%%%%%%%%%%%%%%%%%%%%%%%%%%%%%%%%%%%%%%%%%%%%%%%%%%%%%%%%%%%%%%%%%%%
%% Start Document of the Presentation                                           %%               
%%%%%%%%%%%%%%%%%%%%%%%%%%%%%%%%%%%%%%%%%%%%%%%%%%%%%%%%%%%%%%%%%%%%%%%%%%%%%%%%%%
\begin{document}
\input{libs/code}

%% ---------------------------------------------------------------------------
% Título
\begin{frame}{}
    \includegraphics[scale=0.1]{minerva-ufrj.png} \hspace{3cm} \vspace{-0.2cm}
    \includegraphics[scale=0.15]{poli.png} \hspace{2cm} \vspace{-0.2cm}
    \includegraphics[scale=0.25]{del.png} \hspace{-0.1cm} \vspace{-0.1cm}
    \maketitle
\end{frame} 

%% ---------------------------------------------------------------------------
% Sumário
\begin{frame}{Sumário}
    \begin{multicols}{2}
        \tableofcontents
    \end{multicols}
\end{frame}

%% ---------------------------------------------------------------------------
% SECTION: Motivação
\section{Motivação}

\begin{frame}{Motivação}
    Texto de Exemplo para a motivaçãoasdfasffdsa
\end{frame}


%% ---------------------------------------------------------------------------
% SECTION: Conclusão
\section{Conclusão}

\begin{frame} {Conclusões}
    \begin{itemize}
        \justifying
        \item Em grande parte, os resultados alcançados estão condizentes com os valores esperados.
        \item Discrepância nos valores de T60 podem ser explicados pelas diferenças de implementação entre este projeto e \cite{RIR_Data_Aug}.
        \item Avaliação empírica das sensações subjetivas de “distância” e “eco” condizentes com as modificações esperadas.
    \end{itemize}
\end{frame}

\begin{frame} {Trabalhos Futuros}
    \begin{itemize}
        \justifying
        \item Implementação de uma metodologia de \textit{data augmentation} de T60 mais próxima à usada no artigo \cite{RIR_Data_Aug}.
        \item Comparação entre as RIRs geradas com a metodologia implementada e RIRs geradas através de programas de simulação acústicas (RAIOS \cite{RAIOS}).
        \item Proposta de um modelo de rede de \textit{deep learning} para estimação de T60 e DRR em AVCDs para observação da eficácia das RIRs 
        como aprimoradoras do treinamento de redes neurais.
    \end{itemize}
\end{frame}

\begin{frame}
    \centering
    \huge{\textbf{\example{Obrigado!}}}    
\end{frame}

%% ---------------------------------------------------------------------------
% Referências
\begin{frame}%[allowframebreaks]
    \frametitle{Referências}
    \printbibliography
\end{frame}

\end{document}